% Question 10

\begin{itemize}
    \item[(a)] We have the following output from calling \verb|?Boston|:
    \begin{verbatim}
Boston                  package:ISLR2                  R Documentation

Boston Data

Description:

     A data set containing housing values in 506 suburbs of Boston.

Usage:

     Boston

Format:

     A data frame with 506 rows and 13 variables.

     ‘crim’ per capita crime rate by town.

     ‘zn’ proportion of residential land zoned for lots over 25,000
          sq.ft.

     ‘indus’ proportion of non-retail business acres per town.

     ‘chas’ Charles River dummy variable (= 1 if tract bounds river; 0
          otherwise).

     ‘nox’ nitrogen oxides concentration (parts per 10 million).

     ‘rm’ average number of rooms per dwelling.

     ‘age’ proportion of owner-occupied units built prior to 1940.

     ‘dis’ weighted mean of distances to five Boston employment
          centres.

     ‘rad’ index of accessibility to radial highways.

     ‘tax’ full-value property-tax rate per $10,000.

     ‘ptratio’ pupil-teacher ratio by town.

     ‘lstat’ lower status of the population (percent).

     ‘medv’ median value of owner-occupied homes in $1000s.
    \end{verbatim}
    \item[(b)] In Figure 7, we have some of the data from the \verb|Boston| data
    set displayed in scatterplots.\par
    \begin{figure}[!ht]
        \includegraphics[scale=0.6, center]{./plots/ex2_10_b.pdf}
        \caption{Scatterplots of Boston data}
    \end{figure}
    \qquad From the scatterplots in this figure, we may see that as the distance to 
    employment centres increases, the nitrogen oxide concentration decreases,
    perhaps indicating higher levels of nitrogen oxide in the inner city due
    to vehicles and industrial pollution, which coincides with where employment
    centres are located. Likewise, we see that owner-occupied units built prior
    to the year 1940 are concentrated in the city centre.\par
    \qquad We also see that a higher number of rooms and value of owner-occupied 
    homes tend to be concentrated amongst the lower status of the population.
    \item[(c)] We can see that per capita crime rates are correlated with the
    lower status of the population by looking at Figure 8.
    \begin{figure}[!ht]
        \includegraphics[scale=0.6, center]{./plots/ex2_10_c.pdf}
        \caption{Scatterplot of crim vs lstat data}
    \end{figure}
    It appears that high per capita crime rates are associated with a 
    larger population in the lower status category.
    \item[(d)] Yes, we have the following summary:
    \begin{verbatim}
> summary(Boston$crim)
    Min.  1st Qu.   Median     Mean  3rd Qu.     Max.
 0.00632  0.08204  0.25651  3.61352  3.67708 88.97620
    \end{verbatim}
    While the median and mean for the \verb|crim| data are 
    0.25651 and 3.61352, respectively, the range is $[0.00632, 88.97620]$,
    so that some census tracts must have particularly high crime rates.
    Tax rates appear to be more evenly distributed, as can be seen from the
    following:
    \begin{verbatim}
> summary(Boston$tax)
   Min. 1st Qu.  Median    Mean 3rd Qu.    Max.
  187.0   279.0   330.0   408.2   666.0   711.0
    \end{verbatim}
    The same applies to pupil-teacher ratios:
    \begin{verbatim}
> summary(Boston$ptratio)
   Min. 1st Qu.  Median    Mean 3rd Qu.    Max.
  12.60   17.40   19.05   18.46   20.20   22.00
    \end{verbatim}
    \item[(e)]
    We have the following:
    \begin{verbatim}
> sum(Boston$chas)
[1] 35
    \end{verbatim}
    \item[(f)]
    We have the following:
    \begin{verbatim}
> median(Boston$ptratio)
[1] 19.05
    \end{verbatim}
    \item[(g)]
    The tract corresponding to observation $i = 399$, for which we have
    \begin{verbatim}
> median(Boston$ptratio)
[1] 19.05
    \end{verbatim}
    For that census tract, we have the following data:
    \begin{verbatim}
> Boston[399,]
       crim zn indus chas   nox    rm age    dis rad tax ptratio lstat medv
399 38.3518  0  18.1    0 0.693 5.453 100 1.4896  24 666    20.2 30.59    5
    \end{verbatim}
    We can see that the \verb|crim| variable is far higher than the median 
    value of 3.61352, and is also in the higher range for the variables
    \verb|tax| and \verb|ptratio| for which the range was determined in (d).
    This would likely indicate a correlation between these variables.
    \item[(h)]
    We have the following code to find the number of census tracts which
    average more than 7 and 8 rooms per dwelling:
    \begin{verbatim}
> sum(Boston$rm > 7)
[1] 64
> sum(Boston$rm > 8)
[1] 13
    \end{verbatim}
    We have the following summary of all census tracts averaging more than 8
    rooms:
    \begin{verbatim}
> summary(Boston[Boston$rm > 8, ])
      crim               zn            indus             chas
 Min.   :0.02009   Min.   : 0.00   Min.   : 2.680   Min.   :0.0000
 1st Qu.:0.33147   1st Qu.: 0.00   1st Qu.: 3.970   1st Qu.:0.0000
 Median :0.52014   Median : 0.00   Median : 6.200   Median :0.0000
 Mean   :0.71879   Mean   :13.62   Mean   : 7.078   Mean   :0.1538
 3rd Qu.:0.57834   3rd Qu.:20.00   3rd Qu.: 6.200   3rd Qu.:0.0000
 Max.   :3.47428   Max.   :95.00   Max.   :19.580   Max.   :1.0000
      nox               rm             age             dis
 Min.   :0.4161   Min.   :8.034   Min.   : 8.40   Min.   :1.801
 1st Qu.:0.5040   1st Qu.:8.247   1st Qu.:70.40   1st Qu.:2.288
 Median :0.5070   Median :8.297   Median :78.30   Median :2.894
 Mean   :0.5392   Mean   :8.349   Mean   :71.54   Mean   :3.430
 3rd Qu.:0.6050   3rd Qu.:8.398   3rd Qu.:86.50   3rd Qu.:3.652
 Max.   :0.7180   Max.   :8.780   Max.   :93.90   Max.   :8.907
      rad              tax           ptratio          lstat           medv
 Min.   : 2.000   Min.   :224.0   Min.   :13.00   Min.   :2.47   Min.   :21.9
 1st Qu.: 5.000   1st Qu.:264.0   1st Qu.:14.70   1st Qu.:3.32   1st Qu.:41.7
 Median : 7.000   Median :307.0   Median :17.40   Median :4.14   Median :48.3
 Mean   : 7.462   Mean   :325.1   Mean   :16.36   Mean   :4.31   Mean   :44.2
 3rd Qu.: 8.000   3rd Qu.:307.0   3rd Qu.:17.40   3rd Qu.:5.12   3rd Qu.:50.0
 Max.   :24.000   Max.   :666.0   Max.   :20.20   Max.   :7.44   Max.   :50.0
    \end{verbatim}
    From the above summary, it becomes clear that census tracts with a large 
    average number of rooms per dwelling have correspondingly low crime rates,
    relatively high tax rates, and relatively high pupil-teacher ratios.
\end{itemize}
