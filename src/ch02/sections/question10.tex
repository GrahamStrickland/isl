% Question 10

\begin{itemize}
    \item[(a)] We have the following output from calling \verb|?Boston|:
    \begin{verbatim}
Boston                  package:ISLR2                  R Documentation

Boston Data

Description:

     A data set containing housing values in 506 suburbs of Boston.

Usage:

     Boston

Format:

     A data frame with 506 rows and 13 variables.

     ‘crim’ per capita crime rate by town.

     ‘zn’ proportion of residential land zoned for lots over 25,000
          sq.ft.

     ‘indus’ proportion of non-retail business acres per town.

     ‘chas’ Charles River dummy variable (= 1 if tract bounds river; 0
          otherwise).

     ‘nox’ nitrogen oxides concentration (parts per 10 million).

     ‘rm’ average number of rooms per dwelling.

     ‘age’ proportion of owner-occupied units built prior to 1940.

     ‘dis’ weighted mean of distances to five Boston employment
          centres.

     ‘rad’ index of accessibility to radial highways.

     ‘tax’ full-value property-tax rate per $10,000.

     ‘ptratio’ pupil-teacher ratio by town.

     ‘lstat’ lower status of the population (percent).

     ‘medv’ median value of owner-occupied homes in $1000s.
    \end{verbatim}
    \item[(b)] In Figure 7, we have some of the data from the \verb|Boston| data
    set displayed in scatterplots.\par
    \begin{figure}[!ht]
        \includegraphics[scale=0.6, center]{./plots/ex2_10_b.pdf}
        \caption{Scatterplots of Boston data}
    \end{figure}
    \qquad From the scatterplots in this figure, we may see that as the distance to 
    employment centres increases, the nitrogen oxide concentration decreases,
    perhaps indicating higher levels of nitrogen oxide in the inner city due
    to vehicles and industrial pollution, which coincides with where employment
    centres are located. Likewise, we see that owner-occupied units built prior
    to the year 1940 are concentrated in the city centre.\par
    \qquad We also see that a higher number of rooms and value of owner-occupied 
    homes tend to be concentrated amongst the lower status of the population.
\end{itemize}
