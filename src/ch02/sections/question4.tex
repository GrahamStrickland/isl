% Question 4

\begin{itemize}
    \item[(a)] Three real-life applications in which classification might be useful
    include: 
    \begin{itemize}
        \item[(i)] Determining a voter's party preference based upon the results of 
        a census. The response would be a classification of a voter as being likely 
        to vote for one of the possible political parties and the predictors would be 
        features such as age, demographic, or location, which could be quantitative
        or qualitative. The goal in this case would be prediction, since we would 
        most likely want to determine their vote in the next election.
        \item[(ii)] Classification could also be used to determine the medical 
        condition causing certain symptoms via medical imaging analysis. The 
        response would be any of a number of medical conditions and the predictors  
        would be features of the image, like abnormal textures or shapes within 
        images provided by medical scanning. The goal would be inference, given that 
        the condition already exists and we would like to determine its nature.
        \item[(iii)] Another application would be determining a credit score for
        an individual. The response would be a natural number within a pre-determined 
        range and the predictors would be factors such as time taken to repay debts,
        number of credit lines awarded to the individual, and total accumulated debt.
        The goal would be prediction, since the credit score is used to determine 
        the likelihood that the individual will repay their debts on time.
    \end{itemize}
    \item[(b)] Three real-life applications in which regression might be useful
    include:
    \begin{itemize}
        \item[(i)] Attempting to predict the performance of a stock, given its
        past history would be a suitable application for regression. The response
        would be a numerical value indicating the expected price at a certain time
        and the predictors would be past values on a set of times. The goal would
        be prediction, given that we are estimating future performance.
        \item[(ii)] Regression could also be used to determine the probability that 
        a person will purchase a certain item from a retailer, given their past 
        purchase history. The response would be a probability 
        ($\text{Pr}(X) \in [0, 1]$) and the predictors would be factors such as 
        number of past purchases from that same retailer, spending history, credit 
        rating, etc. The goal would also be prediction.
        \item[(iii)] Another application would be determining the levels of a 
        contaminant in a water supply, based upon readings from sources which are
        not directly drawn from the water supply itself, e.g., taps and waste water.
        The response would be a numeric value (say in mg/L) and the predictors would
        be the equivalent values in the other sources. The goal would be inference,
        since we are attempting to determine a current value.
    \end{itemize}
    \item[(c)] Three real-life applications in which cluster analysis might be 
    useful include:
    \begin{itemize}
        \item[(i)] A useful application of cluster analysis would be attempting
        to determine the species of certain related organisms given the degree to
        which they exhibit certain features.
        \item[(ii)] Cluster analysis could also be used to determine whether certain
        geological samples fit within distinct groups based upon their chemical 
        composition.
        \item[(iii)] Another application of cluster analysis could be using 
        segmenting consumers into certain target markets based upon their response
        to marketing surveys.
    \end{itemize}
\end{itemize}
