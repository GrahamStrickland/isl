% Exercise 3.10

\begin{itemize}
    \item[(a)] We have the following output from a multiple regression of \verb|Sales|
        on \verb|Price|, \verb|Urban|, and \verb|US|:
        \scriptsize\begin{verbatim}
> Carseats <- read.csv(
    "../datasets/Carseats.csv", header = T, 
    na.strings = "?", stringsAsFactors = T
  )
> lm.fit <- lm(Sales ~ Price + Urban + US, data = Carseats)
> summary(lm.fit)

Call:
lm(formula = Sales ~ Price + Urban + US, data = Carseats)

Residuals:
    Min      1Q  Median      3Q     Max 
-6.9206 -1.6220 -0.0564  1.5786  7.0581 

Coefficients:
             Estimate Std. Error t value Pr(>|t|)    
(Intercept) 13.043469   0.651012  20.036  < 2e-16 ***
Price       -0.054459   0.005242 -10.389  < 2e-16 ***
UrbanYes    -0.021916   0.271650  -0.081    0.936    
USYes        1.200573   0.259042   4.635 4.86e-06 ***
---
Signif. codes:  0 ‘***’ 0.001 ‘**’ 0.01 ‘*’ 0.05 ‘.’ 0.1 ‘ ’ 1

Residual standard error: 2.472 on 396 degrees of freedom
Multiple R-squared:  0.2393,    Adjusted R-squared:  0.2335 
F-statistic: 41.52 on 3 and 396 DF,  p-value: < 2.2e-16
        \end{verbatim}\normalsize
    \item[(b)] The \verb|Price| coefficent indicates that a decrease of \$1 in price
        results in a sales increase of about 54 carseats. From the qualititative
        variables \verb|Urban| and \verb|US|, we can see that since \verb|Urban| 
        has a negative coefficient and \verb|US| a positive coefficient, both
        corresponding to the dummy variable \verb|Yes|, we can see that a store in
        an urban area is associated with a slight decrease in sales, while a store 
        in the US is associated with an increase in sales.
    \item[(c)] We have the dummy variables 
        \[
            x_{i1} = 
            \begin{cases}
                1 \quad &\text{if $i$th store is in an urban area} \\
                0 \quad &\text{if $i$th store is not in an urban area} \\
            \end{cases}
        \]
        and
        \[
            x_{i2} = 
            \begin{cases}
                1 \quad &\text{if $i$th store is in the US} \\
                0 \quad &\text{if $i$th store is not in the US} \\
            \end{cases},
        \]
        so that we have the regression model
        \begin{equation*}
            \begin{split}
                y_i &= \beta_0 + \beta_1 x_i + \beta_2 x_{i1} + \beta_3 x_{i2} + \epsilon_i \\
                &=
                \begin{cases}
                    \beta_0 + \beta_1x_i + \beta_2 + \beta_3 + \epsilon_i \quad 
                        &\text{if $i$th store is urban in the US} \\
                    \beta_0 + \beta_1x_i + \beta_2 + \epsilon_i \quad 
                        &\text{if $i$th store is urban not in the US} \\
                    \beta_0 + \beta_1x_i + \beta_3 + \epsilon_i \quad 
                        &\text{if $i$th store is not urban in the US} \\
                    \beta_0 + \beta_1x_i + \epsilon_i \quad 
                        &\text{if $i$th store is neither urban nor US} \\
                \end{cases}.
            \end{split}
        \end{equation*}
    \item[(d)] We can reject the null hypothesis for \verb|Price| and \verb|US|, since 
        the associated $p$-values are less than $2^{-16}$.
    \item[(e)] We have the following regression of \verb|Sales| on \verb|Price| and
        \verb|US|:
        \scriptsize\begin{verbatim}
> lm.fit <- lm(Sales ~ Price + US)
> summary(lm.fit)

Call:
lm(formula = Sales ~ Price + US)

Residuals:
    Min      1Q  Median      3Q     Max 
-6.9269 -1.6286 -0.0574  1.5766  7.0515 

Coefficients:
            Estimate Std. Error t value Pr(>|t|)    
(Intercept) 13.03079    0.63098  20.652  < 2e-16 ***
Price       -0.05448    0.00523 -10.416  < 2e-16 ***
USYes        1.19964    0.25846   4.641 4.71e-06 ***
---
Signif. codes:  0 ‘***’ 0.001 ‘**’ 0.01 ‘*’ 0.05 ‘.’ 0.1 ‘ ’ 1

Residual standard error: 2.469 on 397 degrees of freedom
Multiple R-squared:  0.2393,    Adjusted R-squared:  0.2354 
F-statistic: 62.43 on 2 and 397 DF,  p-value: < 2.2e-16
        \end{verbatim}\normalsize
    \item[(f)] The models fit the data reasonably well, since in both cases, we 
        have a relatively high $F$-statistic and low $p$-values, but the model 
        in (e) is improved by omitting \verb|Urban| as a predictor.
    \item[(g)] We have the following:
        \scriptsize\begin{verbatim}
> confint(lm.fit)
                  2.5 %      97.5 %
(Intercept) 11.79032020 14.27126531
Price       -0.06475984 -0.04419543
USYes        0.69151957  1.70776632
        \end{verbatim}\normalsize
        This indicates that 95\% confidence intervals for the intercept coefficient 
        ($\beta_0$) are $[11.79032020,14.27126531]$, $[-0.06475984, -0.04419543]$ 
        for the \verb|Price| ($\beta_1$) coefficient, and $[0.69151957, 1.70776632]$ 
        for the \verb|USYes| ($\beta_2$) coefficient, in the model
        \[
            f(X) = \beta_0 + \beta_1\verb|Price| + \beta_2\verb|USYes|.
        \]
    \item[(h)] 
\end{itemize}

