% Exercise 3.8

\begin{itemize}
    \item[(a)] We have the following output:
        \begin{verbatim}
> summary(lm.fit)

Call:
lm(formula = mpg ~ horsepower)

Residuals:
     Min       1Q   Median       3Q      Max 
-13.5710  -3.2592  -0.3435   2.7630  16.9240 

Coefficients:
             Estimate Std. Error t value Pr(>|t|)    
(Intercept) 39.935861   0.717499   55.66   <2e-16 ***
horsepower  -0.157845   0.006446  -24.49   <2e-16 ***
---
Signif. codes:  0 ‘***’ 0.001 ‘**’ 0.01 ‘*’ 0.05 ‘.’ 0.1 ‘ ’ 1

Residual standard error: 4.906 on 390 degrees of freedom
  (5 observations deleted due to missingness)
Multiple R-squared:  0.6059,    Adjusted R-squared:  0.6049 
F-statistic: 599.7 on 1 and 390 DF,  p-value: < 2.2e-16
        \end{verbatim}
        \begin{itemize}
            \item[i.] Yes, it is safe to conclude that there is a relationship 
                between the predictor and the response. We can see that from
                the $p$-values for \verb|(Intercept)| and \verb|horsepower|,
                which are very small, i.e., $p < 2.2 \times {10}^{16}$, we may 
                reject the null hypothesis.
            \item[ii.] 
            \item[iii.] 
            \item[iv.] 
        \end{itemize}
    \item[(b)] 
    \item[(c)] 
\end{itemize}
