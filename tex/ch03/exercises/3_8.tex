% Exercise 3.8

\begin{itemize}
    \item[(a)] We have the following output:
        \small\begin{verbatim}
> summary(lm.fit)

Call:
lm(formula = mpg ~ horsepower)

Residuals:
     Min       1Q   Median       3Q      Max 
-13.5710  -3.2592  -0.3435   2.7630  16.9240 

Coefficients:
             Estimate Std. Error t value Pr(>|t|)    
(Intercept) 39.935861   0.717499   55.66   <2e-16 ***
horsepower  -0.157845   0.006446  -24.49   <2e-16 ***
---
Signif. codes:  0 ‘***’ 0.001 ‘**’ 0.01 ‘*’ 0.05 ‘.’ 0.1 ‘ ’ 1

Residual standard error: 4.906 on 390 degrees of freedom
  (5 observations deleted due to missingness)
Multiple R-squared:  0.6059,    Adjusted R-squared:  0.6049 
F-statistic: 599.7 on 1 and 390 DF,  p-value: < 2.2e-16
        \end{verbatim}\normalsize
        \begin{itemize}
            \item[i.] Yes, it is safe to conclude that there is a relationship 
                between the predictor and the response. We can see that from
                the $p$-values for \verb|(Intercept)| and \verb|horsepower|,
                which are very small, i.e., $p < 2.2 \times {10}^{16}$, we may 
                reject the null hypothesis.
            \item[ii.] Since we have multiple $R^2$ and adjusted $R^2$ statistics
                of 0.6059 and 0.6049, respectively, we conclude that close to 2/3
                of the variability in the response can be explained by the regression.
                From this, we conclude that the linear regression is a relatively 
                good fit, and the relationship between the predictor and response
                is strong.
            \item[iii.] Since \verb|mpg| decreases as \verb|horsepower| increases
                along the regression line, we see that the relationship between
                the predictor and response is negative.
            \item[iv.] We have the following output:
                \scriptsize\begin{verbatim}
> predict(lm.fit, data.frame(horsepower = 98), interval = "confidence")
       fit      lwr      upr
1 24.46708 23.97308 24.96108
                \end{verbatim}\normalsize
                from which we conclude that the associated 95\% confidence interval
                is [23.97308, 24.96108] for a predicted value of 24.46708.
                For the prediction interval, we have
                \scriptsize\begin{verbatim}
> predict(lm.fit, data.frame(horsepower = 98), interval = "prediction")
       fit     lwr      upr
1 24.46708 14.8094 34.12476
                \end{verbatim}\normalsize
                so that the interval is [14.8094, 34.12476] for the same predicted
                value.
        \end{itemize}
    \item[(b)] In Figure~\ref{fig9}, we have a plot of the response and predictor,
        as well as the regression line using the \verb|abline()| function.
        \begin{figure}[!ht]
            \includegraphics[scale=0.6, center]{../plots/ex3_8_b.pdf}
            \caption{Response, predictor, and regression line for a linear regression
            of \\ \texttt{mpg} on \texttt{horsepower}\label{fig9}}
        \end{figure}
    \item[(c)] In Figure~\ref{fig10}, we have a plot of the default diagnostics
        for the regression of \verb|mpg| on \verb|horsepower|.
        \begin{figure}[!ht]
            \includegraphics[scale=0.6, center]{../plots/ex3_8_c.pdf}
            \caption{Diagnostic plots for a linear regression of \texttt{mpg} on 
            \texttt{horsepower}\label{fig10}}
        \end{figure}
        We can see from these plots that there are 3 high leverage points corresponding
        to observations 323, 330, and 334. From the residual plots, we can also see 
        the data follows a very clear pattern, bringing into question the conclusions
        of 3.8(a). From this we may conclude that there are some problems with applying
        a linear model to this data.
        
\end{itemize}
