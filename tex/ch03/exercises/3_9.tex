% Exercise 3.9

\begin{itemize}
    \item[(a)] In Figure~\ref{fig3_9scat}, we have a scatterplot matrix of all the 
        \verb|Auto| data.
        \begin{figure}[!ht]
            \includegraphics[scale=0.6, center]{../plots/ex3_9_a.pdf}
            \caption{Scatterplot matrix of \texttt{Auto} data \label{fig3_9scat}}
        \end{figure}
    \item[(b)] We have the following output from using the \verb|cor()| function:
        \scriptsize\begin{verbatim}
> cor(Auto[,1:8])
                    mpg  cylinders displacement horsepower     weight
mpg           1.0000000 -0.7776175   -0.8051269 -0.7784268 -0.8322442
cylinders    -0.7776175  1.0000000    0.9508233  0.8429834  0.8975273
displacement -0.8051269  0.9508233    1.0000000  0.8972570  0.9329944
horsepower   -0.7784268  0.8429834    0.8972570  1.0000000  0.8645377
weight       -0.8322442  0.8975273    0.9329944  0.8645377  1.0000000
acceleration  0.4233285 -0.5046834   -0.5438005 -0.6891955 -0.4168392
year          0.5805410 -0.3456474   -0.3698552 -0.4163615 -0.3091199
origin        0.5652088 -0.5689316   -0.6145351 -0.4551715 -0.5850054
             acceleration       year     origin
mpg             0.4233285  0.5805410  0.5652088
cylinders      -0.5046834 -0.3456474 -0.5689316
displacement   -0.5438005 -0.3698552 -0.6145351
horsepower     -0.6891955 -0.4163615 -0.4551715
weight         -0.4168392 -0.3091199 -0.5850054
acceleration    1.0000000  0.2903161  0.2127458
year            0.2903161  1.0000000  0.1815277
origin          0.2127458  0.1815277  1.0000000
        \end{verbatim}\normalsize
    \item[(c)] We have the following output from using the \verb|lm()| function:
        \small\begin{verbatim}
> lm.fit <- lm(mpg ~ ., data = Auto[,1:8])
> summary(lm.fit)

Call:
lm(formula = mpg ~ ., data = Auto[, 1:8])

Residuals:
    Min      1Q  Median      3Q     Max 
-9.5903 -2.1565 -0.1169  1.8690 13.0604 

Coefficients:
               Estimate Std. Error t value Pr(>|t|)    
(Intercept)  -17.218435   4.644294  -3.707  0.00024 ***
cylinders     -0.493376   0.323282  -1.526  0.12780    
displacement   0.019896   0.007515   2.647  0.00844 ** 
horsepower    -0.016951   0.013787  -1.230  0.21963    
weight        -0.006474   0.000652  -9.929  < 2e-16 ***
acceleration   0.080576   0.098845   0.815  0.41548    
year           0.750773   0.050973  14.729  < 2e-16 ***
origin         1.426141   0.278136   5.127 4.67e-07 ***
---
Signif. codes:  0 ‘***’ 0.001 ‘**’ 0.01 ‘*’ 0.05 ‘.’ 0.1 ‘ ’ 1

Residual standard error: 3.328 on 384 degrees of freedom
Multiple R-squared:  0.8215,    Adjusted R-squared:  0.8182 
F-statistic: 252.4 on 7 and 384 DF,  p-value: < 2.2e-16
        \end{verbatim}\normalsize
        \begin{itemize}
            \item[(i)] Yes, since we have several predictors with low $p$-values
                and/or low standard error relative to the estimate.
            \item[(ii)] The predictors with the most statistically significant
                relationship to the response appear to be \verb|displacement|, 
                \verb|weight|, \verb|year|, and \verb|origin|.
            \item[(iii)] The coefficient of the \verb|year| variable is given
                by 0.750773, suggesting that an increase of 1 year in the model's 
                date of manufacture is associated with an increase of 0.750773 miles
                per gallon in fuel consumption by the vehicle.
        \end{itemize}
    \item[(d)] In Figure~\ref{fig3_9diag}, we have some dianostic plots of the linear 
        regression fit for the \verb|Auto| data with \verb|mpg| as response and all 
        quantitative variables as the predictors.
        \begin{figure}[!ht]
            \includegraphics[scale=0.6, center]{../plots/ex3_9_d.pdf}
            \caption{Diagnostic plots for regression of \texttt{Auto} data \label{fig3_9diag}}
        \end{figure}
        We can see that the residuals vs fitted plot shows a well-defined shape to the
        data that may suggest a non-linear relationship. We can see some high-leverage
        observations in the residuals vs leverage plot and the scale-location plot, 
        with observations 323, 327, and 394 standing out most.
    \item[(e)]
    \item[(f)]
\end{itemize}
