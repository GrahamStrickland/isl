% Exercise 2.9

\begin{itemize}
    \item[(a)] The quantitative predictors are \verb|mpg|, \verb|cylinders|,
    \verb|displacement|, \par \verb|horsepower|, \verb|weight|, \verb|acceleration|,
    \verb|year|, and \verb|origin|. The only qualitative predictor is \verb|name|.
    \item[(b)] The range of each quantitative predictor is as follows:
    \begin{center}
        \begin{tabular}{ l | r | r }
            \multicolumn{1}{c}{Predictor} & \multicolumn{1}{c}{Minimum} 
            & \multicolumn{1}{c}{Maximum} \\
            \hline
            \verb|mpg| & 9.0 & 46.6 \\
            \verb|cylinders| & 3 & 8 \\
            \verb|displacement| & 68 & 455 \\
            \verb|horsepower| & 46 & 230 \\
            \verb|weight| & 1649 & 4997 \\
            \verb|acceleration| & 8.0 & 24.8 \\
            \verb|year| & 70 & 82 \\
            \verb|origin| & 1 & 3
        \end{tabular}
    \end{center}
    \item[(c)] We have the following values for the mean and standard deviation:
    \begin{center}
        \begin{tabular}{ l | r | r }
            \multicolumn{1}{c}{Predictor} & \multicolumn{1}{c}{Mean} 
            & \multicolumn{1}{c}{Standard Deviation} \\
            \hline
            \verb|mpg| & 23.44592 & 7.805007 \\
            \verb|cylinders| & 5.471939 & 1.705783 \\
            \verb|displacement| & 194.412 & 104.644 \\
            \verb|horsepower| & 104.4694 & 38.49116 \\
            \verb|weight| & 2977.584 & 849.4026 \\
            \verb|acceleration| & 15.54133 & 2.758864 \\
            \verb|year| & 75.97959 & 3.683737 \\
            \verb|origin| & 1.576531 & 0.8055182
        \end{tabular}
    \end{center}
    \item[(d)] We have the following adjusted values for the range, mean, and 
    standard deviation:

    \begin{center}
        \begin{tabular}{ l | r | r | r | r }
            \multicolumn{1}{c}{Predictor} & \multicolumn{1}{c}{Min.} 
            & \multicolumn{1}{c}{Max.} & \multicolumn{1}{c}{Mean} 
            & \multicolumn{1}{c}{S.D.} \\
            \hline
            \verb|mpg| & 11.0 & 46.6 & 24.40443 & 7.867283 \\
            \verb|cylinders| & 3 & 8 & 5.373418 & 1.654179 \\
            \verb|displacement| & 68 & 455 & 187.2405 & 99.67837 \\
            \verb|horsepower| & 46 & 230 & 100.7215 & 35.70885 \\
            \verb|weight| & 1649 & 4997 & 2935.972 & 811.3002 \\
            \verb|acceleration| & 8.5 & 24.8 & 15.7269 & 2.693721 \\
            \verb|year| & 70 & 82 & 77.14557 & 3.106217 \\
            \verb|origin| & 1 & 3 & 1.601266 & 0.81991
        \end{tabular}
    \end{center}

    \item[(e)]
    In Figure~\ref{fig6}, we have plots of the \verb|Auto| data highlighting the 
    relationships between some of the variables.
    \begin{figure}[!ht]
        \includegraphics[width=\textwidth, center]{../plots/ex2_9_e.pdf}
        \caption{Scatterplots of Auto data \label{fig6}}
    \end{figure}
    As can be seen by these plots, there is a strong correlation between the
    number of cylinders and the gas mileage (\verb|mpg|)of the vehicle in question.
    This relationship is somewhat the inverse of that between cylinders and
    horsepower. We can also see that mpg and horsepower as well and, to a lesser
    extent, acceleration and horsepower are correlated.
    \item[(f)]
    Yes, we can use both the number of cylinders and the horsepower to estimate
    the gas mileage of the vehicle, since these two variables have a strongly
    correlated relationship to gas mileage.
\end{itemize}
